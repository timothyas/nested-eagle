\title{Nested-EAGLE: \\[.5ex]
    A Data Driven, Global \\[.5ex]
    Weather Model with \\[.5ex]
    High Resolution over \\[.5ex]
    the Contiguous US
}

\author{Timothy A. Smith, NOAA PSL}
\author{Mariah Pope, EPIC}
\author{Sergey Frolov, NOAA PSL}
\author{Brett Basarab, CIRES / NOAA PSL}
\author{Daniel Abdi, CIRES / NOAA GSL}
\author{Isidora Jankov, NOAA GSL}

\section{Goal}

\vspace{-1ex}
Develop a global medium range weather prediction model that:
\vspace{.5ex}

\begin{itemize}
    \item captures synoptic scale forcings
    \item represents precipitation at scale useful for decision makers
    \item produces forecasts at a low computational expense
\end{itemize}

\section{Data}

\vspace{-1ex}
\begin{itemize}
    \item Train on GFS + HRRR ``Analysis'' (fhr=0) for all variables, except
        precipitation, which uses 0-6h forecast accumulations
        \vspace{.25ex}
    \item Implement nested or ``stretched'' grid approach, following Met Norway,
        where HRRR grid is cut out of GFS grid (no overlap)
        \vspace{.25ex}
    \item Use full archives available on NCAR RDA and AWS:
        \begin{itemize}
            \item Training: Feb 2015-Jan 2023
            \item Validation: Feb 2023-Jan 2024
            \item Testing Feb 2024-Jan 2025
        \end{itemize}
        \vspace{.25ex}
    \item Conservatively regrid archived 0.25$^\circ$ GFS to 1.00$^\circ$ and
        3km HRRR to 15km
        \vspace{.25ex}
\end{itemize}

\section{Design Choices that Mattered}

\vspace{-1ex}
\begin{itemize}
    \item Using shifted window processor removed GFS/HRRR boundary artifacts
        \vspace{.25ex}
    \item Reduce CONUS loss weight 50\%$\rightarrow$10\% improved skill
        significantly
        \vspace{.25ex}
\end{itemize}

%% This fills the space between the content and the logo
%\vfill
